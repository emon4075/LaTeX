\documentclass{article}
\pagestyle{empty}
\usepackage{graphicx} % Required for inserting images
\usepackage{amsmath,amsfonts,amssymb}
\usepackage{float}

\begin{document}

The distributive property states that $a(b+c)=ab+ac$, for all $a,b,c \in \mathbb{R}$.\\[6pt]
The equivalence class of $a$ is $[a]$\\[6pt]
The set $A$ is defined to be $\{1,2,3\}$ \\[6pt]
I have a money of $100\$$
$$2(\frac{1}{X^2-1})$$
$2(\frac{1}{X^2-1})$
$$2\left(\frac{1}{X^2-1}\right)$$
$$2\left\{\frac{1}{X^2-1}\right\}$$
$$2\left[\frac{1}{X^2-1}\right]$$
$$2\left \langle \frac{1}{X^2-1}\right \rangle $$
$$2\left | \frac{1}{X^2-1}\right | $$
$$\left. \frac{dy}{dx}\right|_{x=10}$$
$$ \left(\frac{1}{1+\left(\frac{1}{1+x}\right)}\right) $$

Tables: \\[6pt]
\begin{tabular}{|c||c|c|c|c|c|}
    \hline
    $x$ & 1 & 2 & 3 & 4 & 5 \\
    \hline
    $f(x)$ & 10 & 11 & 12 & 13 & 14\\
    \hline
\end{tabular}
\\[1cm]
\begin{table}[H]
\centering
\def\arraystretch{1.6}
\begin{tabular}{|c||c|c|c|c|c|}
    \hline
    $x$ & 1 & 2 & 3 & 4 & 5 \\
    \hline
    $f(x)$ & $\frac{1}{2}$ & 11 & 12 & 13 & 14\\
    \hline
\end{tabular}
\caption{These values represent the values of $f(x)$}
\end{table}
\begin{table}[H]
\caption{The relationship between $f\: \& \:f'$}
\centering
\def\arraystretch{1.6}
\begin{tabular}{|c|c|}
    \hline
    $f(x)$ & $f'(x)$ \\
    \hline
    $x>0$ & The function $f(x)$ is increasing\\
    \hline
\end{tabular}
\end{table}
\begin{table}[H]
\caption{The relationship between $f\: \& \:f'$}
\centering
\def\arraystretch{1.6}
\begin{tabular}{|l|p{3in}|}
    \hline
    $f(x)$ & $f'(x)$ \\
    \hline
    $x>0$ & The function $f(x)$ is increasing.The function $f(x)$ is increasing.The function $f(x)$ is increasing.The function $f(x)$ is increasing.The function $f(x)$ is increasing\\
    \hline
\end{tabular}
\end{table}

Equation Arrays: 
\begin{align} % align is in math mode
    5x^{21} \; \text{Place Your Text Here}\\
     5x^{21}-9=x+3\\
     5x^2-x-12=0\\
\end{align}

\begin{align} % align is in math mode
     5x^{21}-9&=x+3\\ %used & to align the = signs
     5x^2-x-12&=0\\
     &=12x-6-x^2
\end{align}

\begin{align*} % align is in math mode and * will off the numbering
     5x^{21}-9&=x+3\\ %used & to align the = signs
     5x^2-x-12&=0\\
     &=12x-6-x^2
\end{align*}

\begin{align} % align is in math mode
     5x^{21}-9&=x+3\\ %used & to align the = signs
     5x^2-x-12&=0\\
     &=12x-6-x^2
\end{align}
\end{document}
